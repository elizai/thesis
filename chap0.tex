\begin{savequote}[75mm]
Internetul este un amalgam de tehnologii, legate împreună cu bandă adezivă,
sfoară și gumă de mestecat. Nu este ceva proiectat într-un mod elegant,
pentru că este un organism în creștere, nu o mașinărie construită
cu intenție.
\qauthor{Mattias Petter Johansson (Programator la Spotify)}
\end{savequote}

\chapter{Introducere}

Internetul a evoluat continuu și a ajuns în punctul în care poate face o mulțime
de lucruri pentru care nici măcar nu a fost creat. Aproape toți programatorii din
ziua de azi sunt programatori web, iar aplicațiile web seamănă tot mai mult cu 
aplicațiile desktop. În aceste condiții, a devenit foarte important pentru
dezvoltatori să poată crea astfel de aplicații într-un mod rapid și eficient, 
iar uneltele pe care le au la dispoziție au fost reînnoite permanet cu altele
mai bune.

Arhitectura web clasică este una client-server, în care clientul (browserul)
cere o pagină folosind protocolul HTTP, serverul o crează dinamic folosind un 
limbaj de programare server-side (C\#, Java, Python, PHP, Scala etc.) și o trimite 
browserului pentru afișare. Prin HTTP, conexiunile sunt întotdeauna inițiate 
de către client, care cere pagina web.

Această arhitectură este limitată. Să ne imaginăm de exemplu că avem o pagină
web care afișează în timp real scorurile unor partide de fotbal. După încărcarea
paginii, server-ul nu-i poate comunica browserului că un scor s-a schimbat.
Browserul va afișa scorurile neactualizate până când utilizatorul reîmprospătează
pagina.

Această problemă a fost rezolvată prin intermediul
AJAX\footnote{Asynchronous JavaScript and XML; în aplicațiile moderne
se utilizează cu preferință JSON (JavasScript Object Notation) în loc de XML,
dar denumirea a rămas.},
o tehnică ce permite browserului să facă cereri asincrone către server după ce 
pagina a fost încărcată, prin intermediul JavaScript.

Următoarea etapă în acest proces incremental a fost crearea de 
\emph{Single-Page Application}\footnote{\url{http://en.wikipedia.org/wiki/Single-page\_application}},
denumite în continuare SPA. Într-un SPA, tot codul HTML, JavaScript și CSS este
fie descărcat în momentul în care pagina este încărcată prima dată, fie în mod asincron, de obicei ca
răspuns la acțiunile utilizatorului.

SPA oferă utilizatorului senzația unei
aplicații fluide și poate uneori să ofere iluzia că aceasta răspunde la acțiuni
imediat, fără să mai aștepte răspunsul serverului. Vom vedea în aplicația
construită pentru această lucrare, de exemplu, că atunci când utilizatorul
dorește ștergerea unei resurse, această resursă este întâi înlăturată din UI,
apoi o cerere asincronă îi spune serverului să șteargă resursa din baza de date.
Desigur, pentru că se comunică cu serverul prin TCP/IP, această comunicare
poate eșua, caz în care un mesaj de eroare este afișat și resursa reapare în UI,
dar în mai mult de 90\% din cazuri, când totul merge bine, utilizatorul are
senzația că resursa este ștearsă instant.

Două companii foarte mari, Google și Facebook,
au creat fiecare  câte un framework pentru crearea de SPA: AngularJS și React,
confirmând importanța acestui tip de aplicații. Experiență fluidă pentru utilizator, împreună cu alte avantaje pe care le
vom discuta în capitolul următor au făcut ca SPA să crească foarte mult în popularitate
în ultimii ani.



