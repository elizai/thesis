\chapter{Concluzii}

Aplicațiile web SPA sunt pe un trend ascedent și vor deveni tot mai
comune în anii ce urmează. Necesitatea creării rapide de aplicații SPA
a împins companii mari să pună la dispoziția dezvoltatorilor
tehnologii care să le ușureze munca.

Printre avantajele aplicațiilor SPA se numără timpul mic de 
răspuns și o experiență asemănătoare aplicațiilor native.
Ca și dezavantaje avem complexitatea si timpul de dezvoltare.

Scopul proiectului dezvoltat ca parte din această lucrare,
nu este acela de a crea o aplicație plină de funcționalități
și cu o utilitate în lumea reală. Scopul aplicației este
de a evalua și a demonstra anumite tehnologii pentru
crearea de aplicații web. Accentul nu este pus pe
numărul de trăsături ale aplicației, ci pe implementarea corectă
a acestora, dar și pe implementarea ușoară, într-un timp cât mai
scurt. Implementarea rapidă de trăsături pentru o aplicație
este un criteriu foarte important pe baza căruia companiile
angajează programatori.

Denumirea SPA, duce oarecum în eroare, datorită faptului
că implică faptul că aplicația are o singură pagină.
O denumire mult mai descriptivă este cea de 
\emph{Rich Web Application}. Această denumire sugerează,
în mod corect, că scopul unei astfel de aplicații este
defapt de a micșora diferențele existente între aplicațiile
web și aplicațiile desktop.

Pentru crearea de SPA, pe partea de back-end, 
alternativele sunt multe, mature și stabile.
Django este una dintre aceste alternative, fiind o tehnologie
matură ce pune accent pe lizibilitate și dezvoltarea rapidă și 
cu ușurință de aplicații web moderne.

Un lucru foarte important pentru construirea back-end-ului este
folosirea unui API REST. Un astfel de back-end este decuplat
de front-end, permițând refolosirea aceluiași back-end, de exemplu, pentru
aplicații mobile. Django, împreună cu Django REST Framework este
o soluție matură pentru crearea unui back-end într-un timp
scurt.

Pe partea de front-end, situația nu este la fel de clară. AngularJS este
în momentul de față cea mai matură tehnologie, dar, după părerea multora,
această tehnologie nu va mai fi aici peste câțiva ani.
AngularJS 2.0 a fost anunțat deja, iar acesta nu va fi compatibil
cu versiunea actuală, adică migrarea unei aplicații AngularJS scrisă
cu versiunea 1.3 la versiunea 2.0 va fi foarte dificilă. Există și alte
alternative promițătoare, cum ar fi ReactJS sau Meteor, 
dar aceste tehnologii nu sunt la fel de mature ca AngularJS.

Este foarte posibil ca toate tehnologiile folosite în momentul
de față pe front-end pentru aplicații SPA să nu mai fie folosite
în 5 ani. Acest lucru nu face decât să sublinieze necesitatea oricărui
dezvoltator de aplicații web de a fi la curent în permanență
cu tehnologiile folosite și de a învăța permanent pe parcursul
întregii cariere. De asemenea, este important ca un dezvoltator
să nu cadă în capcana de a se concentra prea mult pe tehnologii,
scăpând astfel din vedere faptul că ele sunt doar unelte
pentru rezolvarea unor probleme.

În ciuda problemelor sale, AngularJS este folosit în multe aplicații
puse în producție și reprezintă o alegere relativ sigură
pentru dezvoltarea unei aplicații SPA.

Deși accentul acestei lucrări a fost pus destul de mult pe
tehnologii, nu vreau să las impresia că cea mai importantă
trăsătură a unui dezvoltator de aplicații este faptul că
a lucrat cu anumite tehnologii. Tehnologiile folosite
sunt cele care ne fac mai productivi ca programatori
și uneori ele pot fi cauza unui proiect reușit sau eșuat.
În același timp, este important pentru un dezvoltator
să aibe înțeleagă principiile din spatele unei tehnologii
mai mult decât tehnologia în sine.

De exemplu, consider că un programator care știe
să folosească Django REST Framework, dar nu știe
proprietățile unui API REST, diferențele dintre 
metodele HTTP și când trebuie folosită fiecare
metodă HTTP este mai puțin valoros decât un
dezvoltator care nu a folosit niciodată
Django REST Framework, dar știe foarte
bine toate proprietățile unui API REST.
Un dezvoltator care are cunoștințele fundamentale
bine puse la punct se poate ține la curent
cu ultimele framework-uri mult mai ușor
decât unul ce are lagune în acest domeniu.