\chapter{Concluzii}

Aplicațiile web SPA sunt pe un trend ascedent și vor deveni tot mai
comune în anii ce urmează. Necesitatea creării rapide de aplicații SPA
a împins companii mari să pună la dispoziția dezvoltatorilor
tehnologii care să le ușureze munca.

Pe partea de back-end, alternativele sunt multe, mature și stabile.
Django este una dintre aceste alternative, fiind o tehnologie
matură ce pune accent pe lizibilitate și dezvoltarea rapidă și 
cu ușurință de aplicații web moderne.

Pe partea de front-end, situația nu este la fel de clară. AngularJS este
în momentul de față cea mai matură tehnologie, dar nu este foarte stabilă.
AngularJS 2.0 a fost anunțat deja, iar acesta nu va fi compatibil
cu versiunea actuală, adică migrarea unei aplicații AngularJS scrisă
cu versiunea 1.3 la versiunea 2.0 va fi dificilă. Există și alte
alternative promițătoare, cum ar fi ReactJS sau Meteor, 
dar aceste tehnologii nu sunt la fel de mature ca AngularJS.
Este foarte posibil ca toate tehnologiile folosite în momentul
de față pe front-end pentru aplicații SPA să nu mai fie folosite
în 5 ani.

Acest lucru nu face decât să sublinieze necesitatea oricărui
dezvoltator de aplicații web de a fi la curent în permanență
cu tehnologiile folosite și de a învăța permanent pe parcursul
întregii cariere. De asemenea, este important ca un dezvoltator
să nu cadă în capcana de a se concentra prea mult pe tehnologii,
scăpând astfel din vedere faptul că ele sunt doar unelte
pentru rezolvarea unor probleme.