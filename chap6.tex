\chapter{Detalii de implementare}

\section{Structura de directoare}

Aplicația, la nivelul cel mai înalt, are 3 module Python (interpretorul
Python directoarele ce conțin fișierul gol \texttt{\_\_init\_\_.py} ca
fiind module):
\begin{description}
\item [authentication] este modulul responsabil înregistrare și autentificarea
utilizatorilor. Acest modul conține modelul ORM al utilizatorului și
view-urile REST corespunzătoare creării de utilizatori noi și
autentificării unui utilizator.
\item [expenses] este modulul ce conține modelul ORM al unuei cheltuieli
și view-urile REST corespunzătoare operațiilor CRUD asupra
cheltuielilor.
\item [money\_keep] este modulul responsabil de configurarea aplicației.
Conține, printre altele, fișierul de configurare \texttt{settings.py}
și fișierul ce conține toate rutele aplicației (URL-urile la 
care aplicația răspunde), texttt{urls.py}.
\end{description}

Pe lângă aceste module, la nivelul cel mai înalt al aplicației
mai sunt directoarele:

\begin{description}
\item [static] conține toate resursele statice: 
  \begin{itemize}
  \item componentele instalate cu \emph{Bower}
  \item CSS-urile customizate ale aplicației
  \item întreaga aplicație AngularJS, cu toate șabloanele,
  controllerele, directivele și serviciile ei.
  \end{itemize}
\item [templates] conține șabloanele Django. Ele sunt scrise
folosind sintaxa \emph{Django Template Language} (DTL) și nu trebuie confundate
cu șabloanele din directorul "static/templates". Șabloanele din
acest director sunt folosite de Django, iar cele din "static" sunt
folosite de AngularJS.
\end{description}

\begin{lstlisting}[title=Structura de directoare a aplicației]
|-- authentication
|   |--- migrations
|-- expenses
|   |--- migrations
|-- money_keep
|-- static
|   |-- bower_components
|   |-- css
|   |-- js
|   |   |-- authentication
|   |   |   |-- controllers
|   |   |   |--- services
|   |   |-- expenses
|   |   |   |-- controllers
|   |   |   |-- directives
|   |   |   |--- services
|   |   |-- layout
|   |   |   |--- controllers
|   |   |--- utils
|   |       |-- controllers
|   |       |--- services
|   |--- templates
|       |-- authentication
|       |-- expenses
|       |-- layout
|       |--- utils
|--- templates
\end{lstlisting}


\section{Autentificarea}

Django are un model folosit în mod standar pentru autentificare,
dar în această aplicație am ales să nu folosesc acest model pentru
a avea mai mult control.
Am creat un alt model, numit \texttt{Account}, care derivă din
\texttt{AbstractBaseUser} și am adăugat în \texttt{settings.py}
următoarea linie pentru ca Django să știe că trebuie să folosească
noul model:
\begin{lstlisting}[language=python]
AUTH_USER_MODEL = 'authentication.Account'
\end{lstlisting}

