\chapter{Alte tehnologii folosite în aplicație}

\section{Bootstrap}

\emph{Boostrap}\footnote{\url{http://getbootstrap.com}}, 
dezvoltat inițial de Mark Otto 
și Jacob Thornton la Twitter, este un framework ce simplifică stilizarea
elementelor HTML cum ar fi form-uri și butoane și crearea
de meniuri de navigație, casete modale etc. De asemenea,
o parte foarte importantă a framework-ului este crearea aplicațiilor
\emph{responsive}, ceea ce înseamnă că aplicația este afișată
corect pe dispozitive cu display-uri de dimensiuni diferite:
calculatoare, tablete și telefoane.

Folosirea framework-ului este destul de simplă. Este
suficient să fie adăugate două fișiere din Bootstrap,
apoi clasele CSS pot fi folosite pentru stilizarea
elementelor HTML.


\section{Bower}

\emph{Bower}\footnote{\url{http://bower.io}} este o unealtă
folosită pentru managementul dependențelor de pe front-end.
Bower folosește fișierul \texttt{bower.json} pentru a 
ști ce trebuie să instaleze.

\lstinputlisting[title=app/bower.json]{chap4-code/bower.json}

Dependențele din \texttt{bower.json} sunt descărcate dintr-un
repository central în directorul \texttt{static/bower\_components}
cu comanda \texttt{bower install}.

\section{Git și GitHub}

\emph{Git}\footnote{https://git-scm.com} este un sistem distribuit
pentru controlul sistemului de reviziuni al fișierelor,
scris inițial de creatorul kernel-ului Linux, Linus Torvalds.
GitHub\footnote{http://github.com} este un serviciu
online ce permite găzduirea repository-urilor Git.

Atât codul aplicației, cât și codul aceastei lucrări, redactată folosind \LaTeX{},
sunt găzduite pe GitHub la adresele \url{https://github.com/stefan-mihaila/money-keep}
și \url{https://github.com/stefan-mihaila/thesis}.


