\begin{savequote}[75mm]
Internetul este un amalgam de tehnologii, legate împreună cu bandă adezivă,
sfoară și gumă de mestecat. Nu este ceva proiectat într-un mod elegant,
pentru că este un organism în creștere, nu o mașinărie construită
cu intenție.
\qauthor{Mattias Petter Johansson (Programator la Spotify)}
\end{savequote}

\chapter{Introducere}

Internetul a evoluat continuu și a ajuns în punctul în care poate face o mulțime
de lucruri pentru care nici măcar nu a fost creat. Aproape toți programatorii din
ziua de azi sunt programatori web, iar aplicațiile web seamănă tot mai mult cu 
aplicațiile desktop. În aceste condiții, a devenit foarte important pentru
dezvoltatori să poată crea astfel de aplicații într-un mod rapid și eficient, 
iar uneltele pe care le au la dispoziție au fost reînnoite permanet cu altele
mai bune.

Arhitectura web clasică este una client server, în care clientul (browserul)
cere o pagină folosind protocolul HTTP, serverul o crează dinamic folosind un 
limbaj de programare server-side (C\#, Java, Python, PHP, Scala etc.) și o trimite 
browserului pentru afișare. Prin HTTP, conexiunile sunt întotdeauna inițiate 
de către client, care cere pagina web.

Această arhitectură este limitată. Să ne imaginăm de exemplu că avem o pagină
web care afișează în timp real scorurile unor partide de fotbal. După încărcarea
paginii, server-ul nu-i poate comunica browserului că un scor s-a schimbat.
Browserul va afișa scorurile neactualizate până când utilizatorul reîmprospătează
pagina.

Această problemă a fost rezolvată prin intermediul
AJAX\footnote{Asynchronous JavaScript and XML; în aplicațiile moderne
se utilizează cu preferință JSON (JavasScript Object Notation) în loc de XML,
dar denumirea a rămas.},
o tehnică ce permite browserului să facă cereri asincrone către server după ce 
pagina a fost încărcată, prin intermediul JavaScript.

Următoarea etapă în acest proces incremental a fost SPA - Single-Page Application.
Într-un SPA, tot codule HTML, JavaScript și CSS este fie descărcat în momentul
în care pagina este încărcată prima dată, fie în mod asincron, de obicei ca
răspuns la acțiunile utilizatorului. SPA oferă utilizatorului senzația unei
aplicații fluide și poate uneori să ofere iluzia că aceasta răspunde la acțiuni
imediat, fără să mai aștepte răspunsul serverului. Vom vedea în aplicația
construită pentru această lucrare, de exemplu, că atunci când utilizatorul
dorește ștergerea unei resurse, această resursă este întâi înlăturată din UI,
apoi o cerere asincronă îi spune serverului să șteargă resursa din baza de date.
Desigur, pentru că se comunică cu serverul prin TCP/IP, această comunicare
poate eșua, caz în care un mesaj de eroare este afișat și resursa reapare în UI,
dar în mai mult de 90\% din cazuri, când totul merge bine, utilizatorul are
senzația că resursa este ștearsă instant.

\section{Modalități de dezvoltare de SPA}

\subsection{Framework-uri JavaScript}

Anumite framework-uri JavaScript pentru creare de aplicații web, cum ar fi
AngularJS\footnote{https://angularjs.org/},
Ember.js\footnote{http://emberjs.com/},
React\footnote{http://facebook.github.io/react/} și
Backbone.js\footnote{http://backbonejs.org/},
și-au propus să ușureze dezvoltarea
de aplicații web SPA.

Printre avantajele folosirii unui framework unui astfel
de framework, enumerăm:
\begin{itemize}
  \item Aceste framework-uri oferă de obicei și posibilitatea organizării codului
  folosind MVC\footnote{Model View Controller}. Într-un framework JavaScript MVC,
  view-ul sunt reprezentate de șabloane HTML, controller-ul este un obiect JS
  care se ocupă de comunicarea dintre view și model, iar modelul este un obiect JS
  care, de obicei, mapează obiectele din baza de date de pe server la obiecte
  stocate în client. Un proiect MVC este mai ușor de navigat, este mai ușor
  de modificat și mai ușor de înțeles. De asemenea, colaborarea dintre
  designer și programator este mai ușoară cu un framework MVC.
  \item Viteza de dezvoltare.
  \item Ușurința de dezvoltare.
  \item Diminuarea codului necesar a fi scris.
\end{itemize}

Bineînțeles, există și dezavantaje:
\begin{itemize}
  \item Necesitatea învățării unei tehnologii noi. Ce este și mai trist, este
  că foarte posibil această tehnologie va fi depășită în doar câțiva ani.
  \item Unele framework-uri (AngularJS) au o curbă de învățare abruptă.
\end{itemize}

\subsection{}

